% -*- TeX -*- -*- UK -*- -*- Soft -*-

\chapter{\libraddask{} Setup}
\label{chap:libraddaskSetup}

\section{Set up Python for Dask}

\subsection{Creating the environment}

Create environment with Anaconda with the appropriate packages and versions to use for Dask.
In this case we are using Python 3.7 to run the Dask scheduler and workers on the server.
Exactly the same versions must be installed on all three computers.

\begin{lstlisting}
    conda create --name devpy37 python=3.7 anaconda
\end{lstlisting}

\subsection{Additional packages}
\label{sec:Additionalpackages}

Ensure that the following packages of the same version numbers are installed on all the computers. 
Some or all of these packages may already be present as part of the full Anaconda installation.
The package numbers installed at the current time on our computers are also listed.
Note that this list only covers the minimum Dask requirements, your local application may require other packages as well.
\begin{lstlisting}
    'dask': '2.12.0'
    'distributed': '2.12.0'
    'msgpack': '0.6.1'
    'cloudpickle': '1.3.0'
    'tornado': '6.0.4'
    'toolz': '0.10.0'
    'numpy': '1.18.1'
\end{lstlisting}

Search for packages here: 
\lstinline{https://anaconda.org/anaconda/} for standard Anaconda packages and 
\lstinline{https://anaconda.org/conda-forge/} for community-supported packages.

To see a list of the currently installed packages on a computer type one of the following, to see versions for all package or a specific package:
\begin{lstlisting}
conda list
conda list dask
\end{lstlisting}

To install a specific version of a package use the command, with appropriate replacement of the package name and version number:
\begin{lstlisting}
conda install package=version

conda install dask=2.12.0
conda install distributed=2.12.0
\end{lstlisting}



------------------------------------------------------------------------------------------------------

\subsection{Make Morticia known}

The morticia library is not installed in the Python site-packages tree, so Python must know where to find it. To tell Python where to find this library create a file with the filename \lstinline{morticia.pth} in the \lstinline{site-packages} folder of the \lstinline{mort} environment. On my PC the environment is here: 

    C:/ProgramData/Anaconda3/envs/mort/lib/site-packages/morticia.pth

On Linux it is in a location similar to this

    ~/anaconda2/envs/mordevpy37/lib/python3.7/site-packages/morticia.pth

The file must have only one line, and this line must be the morticia library location.    In my case the library was cloned from the repository into the following folder, and this must be the contents of the file:

    W:/Morticia/OSS-gitlab/morticia

or on the nimbus server (146.64.246.94) the contents must be

    /home/dgriffith/GitHub/morticia



\subsection{Prepare for Jupyter}

Make the environment visible in a Jupyter notebook"

    conda install notebook ipykernel
    ipython kernel install

See [here])https://github.com/NelisW/ComputationalRadiometry/blob/master/00-Installing-Python-pyradi.ipynb) for more detail.



\section{Work on nimbus (146.64.246.94) for user \lstinline{dgriffith}}

Home directory \lstinline{~} is \lstinline{/home/dgriffith}

\subsection{Bring nimbus to Python 3 status}

Installed a python 3 environment for morticia: mordevpy37, with the morticia dependencies listed above.

Changed the folder name  \lstinline{~/GitHub/morticia} to \lstinline{~/GitHub/morticia2} to keep the old copy available.

Cloned the python 3 version of morticia into folder \lstinline{~/GitHub/morticia}, from \lstinline{https://github.com/NelisW/morticia}

To see where the \lstinline{site-packages} folder on your anaconda installation are use this code:

\begin{lstlisting}
    from distutils.sysconfig import get_python_lib
    print(get_python_lib())

\end{lstlisting}
Added \lstinline{morticia.pth} to the prepared environment's \lstinline{site-packages} folder, e.g.,  \lstinline{~/anaconda2/envs/mordevpy37/lib/python3.7/site-packages/morticia.pth}. The contents of this file is the single line. 

    \lstinline{/home/dgriffith/GitHub/morticia}


\subsection{libRadTran server}

1. On the PC activate the Python 3 morticia environment 

1. Start up the VPN (if used) and find the IP number allocated to the VPN ethernet0. On Windows uUse \lstinline{ipconfig} and look for something like the following (the Ethernet adapter connection number may be different, look for the DNS suffix that says \lstinline{csir.co.za}). This should be an IP address on the CSIR network \lstinline{146.64.xxx.xxx}:

\begin{lstlisting}
Ethernet adapter Local Area Connection 2:

Connection-specific DNS Suffix  . : csir.co.za
Link-local IPv6 Address . . . . . : fe80::8992:152a:c1f0:1260%27
IPv4 Address. . . . . . . . . . . : 146.64.202.118
Subnet Mask . . . . . . . . . . . : 255.255.0.0
Default Gateway . . . . . . . . . :
\end{lstlisting}

    To find the IP address on Linux use \lstinline{ip addr show}.


1. On the PC start the Dask scheduler, using the VPN IP address


\begin{lstlisting}
      dask-scheduler --host 146.64.202.118
\end{lstlisting}

1. On nimbus activate the Python 3 morticia environment 

1. On nimbus cd to the libRadTran bin folder  \lstinline{/home/dgriffith/libRadtran/libRadtran-2.0.2/bin/}

1. On nimbus start the Dask scheworker process, using the VPN IP address and the Dask port number

\begin{lstlisting}
    dask-worker tcp://146.64.202.118:8786
\end{lstlisting}

1. Open the notebook and set the scheduler using the VPN host address



