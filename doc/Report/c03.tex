% -*- TeX -*- -*- UK -*- -*- Soft -*-


\chapter{\libradtran{}}
\label{chap:libRadtran}


\section{\libradtran{} Overview}

From \cite{EmdeLibRadtran2016}:
``\libradtran{} is a widely used software package for
radiative transfer calculations. It allows one to compute (polarized) radiances, irradiance, and actinic fluxes in the solar and thermal spectral regions. \libradtran{} has been used
for various applications, including remote sensing of clouds,
aerosols and trace gases in the Earth's atmosphere, climate
studies, e.g., for the calculation of radiative forcing due to
different atmospheric components, for UV forecasting, the
calculation of photolysis frequencies, and for remote sensing
of other planets in our solar system. The package has been
described in Mayer and Kylling (2005) \cite{libRadtran2005}. Since then several
new features have been included, for example polarization,
Raman scattering, a new molecular gas absorption parameterisation, and several new parameterizations of cloud and
aerosol optical properties. Furthermore, a graphical user interface is now available, which greatly simplifies the usage
of the model, especially for new users. This paper gives an
overview of \libradtran{} version 2.0.1 with a focus on new features. Applications including these new features are provided
as examples of use. A complete description of \libradtran{} and
all its input options is given in the user manual included in
the \libradtran{} software package, which is freely available at
\lstinline{http://www.libradtran.org}.''

See also 
\cite{libRadtran2005,libRadTranUserGuide2012,libRadtranDownload2020,libRadtranmuenchen2020}.


\section{Obtaining and Building \libradtran{}}

\section{Reference Systems}
For reference purposes this installation was done on the following Linux systems.

\subsubsection{Nimbus Server}
This is an existing server that had \libradtran{} 2.0.0 running on it before, so all the pre-requisites were installed.  

\begin{lstlisting}[style=tinysize]
(mordevpy27) dgriffith@nimbus:~/libRadtran/libRadtran-2.0.3/examples$ uname -a
Linux nimbus 4.9.0-8-amd64 #1 SMP Debian 4.9.110-3+deb9u6 (2018-10-08) x86_64 GNU/Linux
\end{lstlisting}

\subsubsection{Fresh Install}

This was a newly-set-up Ubuntu 20.04 installation, in an Oracle VirtualBox (Ver 6.1.6).  This means that none of the prequisites were already installed on the system.

\begin{lstlisting}[style=tinysize]
(base) nwillers@nwillers-VirtualBox:~$ uname -a 
Linux nwillers-VirtualBox 5.4.0-26-generic #30-Ubuntu SMP Mon Apr 20 16:58:30 UTC 2020 x86_64 x86_64 x86_64 GNU/Linux
\end{lstlisting}

The initial installation (for other general development work) was as follows. The packages were installed using synaptic, my preferred Ubuntu installer.

\begin{enumerate}
\itemsep-0.5em 
\item CMake.
\item Make (4.2.1-1.2).
\item GCC (4:9.3.0-1ubuntu2).
\item Perl (might be installed already).
\item gparted.
\item Konsole.
\item Kdiff.
\item git.
\end{enumerate}

\libradtran{} requires additional tools: \lstinline{http://www.libradtran.org/doku.php?id=download}.
Don't take shortcuts, it does not work. Take the trouble to install these (in the Ubuntu package name convention):

\begin{enumerate}
\itemsep-0.5em 
\item flex.
\item gfortran.
\item NetCDF for C: libnetcdf-dev.
\item NetCDF for Fortran: libnetcdff-dev.
\item The GNU Scientific Library: gsl-bin, libgsl-dev, libgsl23 libglscblas0.
\end{enumerate}


\subsection{Downloading  \libradtran{}}

Download \libradtran{} and the optional additional files/modules from\\
\lstinline{http://www.libradtran.org/doku.php?id=download}\\
For our application the REPTRAN absorption parameterization data was required, but it may differ between different sites.




\subsection{Create the server account and software}

\begin{enumerate}
\item Create a user account on the server from which libRadtran is served. This is a standard Linux admin task.  For the present case assume the account name is \lstinline{dgriffith}.

\item Download libRadtran from \lstinline{http://www.libradtran.org/doku.php?id=download} and follow the installation instructions. The instructions are repeated below.      In this case \libradtran{} version 2.0.3 is downloaded and installed.

\item Copy the downloaded file to the folder where it must be installed. Unzipping the compressed tar file with the following command will create a folder \lstinline{libRadtran-2.0.3}

\begin{lstlisting}
    tar -xvf  libradtran-2.0.3.tar.gz
\end{lstlisting}
\item The \libradtran{} installation, building, and testing requires Python 2.7 (installation will fail if Python 3 is used).  If necessary, install Python 2.7 in a conda environment in order to proceed with the \libradtran{} installation.  For the purpose of this discussion we assume that the Python 2.7 environment is named \lstinline{devpy27}.

\begin{lstlisting}
    conda create --name devpy27 python=2.7 anaconda
\end{lstlisting}

\item Compile the distribution by doing the following:

\begin{lstlisting}
    conda activate devpy27
    cd libRadtran-2.0.3
    ./configure
    make
\end{lstlisting}
\item Test the program, (make sure to use GNU make):

\begin{lstlisting}
    conda activate devpy27
    make check 
\end{lstlisting}
 As the test progresses the test results will display in the terminal (the display shown here has been shortened):
\begin{lstlisting}[style=tinysize]
(base) dgriffith@nimbus:~/libRadtran/libRadtran-2.0.3$ conda activate devpy27
(devpy27) dgriffith@snimbus:~/libRadtran/libRadtran-2.0.3$ make check
for dir in examples src_py libsrc_c libsrc_f src GUI lib; do make -C $dir all || exit $?; done
make[1]: Entering directory '/home/dgriffith/libRadtran/libRadtran-2.0.3/examples'

...

make[1]: Entering directory '/home/dgriffith/libRadtran/libRadtran-2.0.3/test'
/usr/bin/perl test.pl
Running various libRadtran tests. This may take some time....

The numbers in the parenthesis behind the name of the tests are:
 1st number:  The lower absolute limit of values included in the test.
              Values in the output less than limit are ignored.
 2nd number:  The maximum difference allowed between local test
              results and the standard results (in percentage).

If this is still unclear, check the source in test/test.pl.in.

make_slitfunction test
make_slitfunction (0.00001,   0.1)............................................. ok.
All make_slitfunction tests succeeded.
make_angresfunc test
make_angresfunc (0.00001,   0.1)............................................... ok.
All make_angresfunc tests succeeded.
angres test
angres (0.00001,   0.1)........................................................ ok.
All angres tests succeeded.
Some \lstinline{uvspec} tests
uvspec simple (0.00001,   0.1)................................................. ok.
disort clear sky (0.00001,   0.1).............................................. ok.
disort aerosol (0.00001,   0.1)................................................ ok.
disort aerosol moments (0.00001,   0.1)........................................ ok.
disort aerosol refractive index (0.00001,   0.1)............................... ok.
disort BRDF Ross-Li (0.00050,   0.1)........................................... ok.
disort water cloud (0.00001,   0.1)............................................ ok.
disort SO2 (0.00001,   0.1).................................................... ok.
disort radiances (0.00001,   0.3).............................................. ok.
disort SCIAMACHY HG approximation (0.00001,   0.3)............................. ok.
disort aerosol (0.00001,   0.1)................................................ ok.
disort wc Legendre moments (0.00001,   0.1).................................... ok.
disort water and ice clouds (0.00001,   0.2)................................... ok.
disort cloud overlap random (0.00001,   0.2)................................... ok.
disort cloud overlap maximum-random (0.00001,   0.2)........................... ok.
disort cloud overlap maximum (0.00001,   0.2).................................. ok.
disort, albedo and altitude map (0.00001,   0.7)............................... ok.
disort BRDF (0.00050,   0.1)................................................... ok.
disort BRDF Hapke (0.00050,   0.1)............................................. ok.
disort BRDF AMBRALS (0.00050,   0.1)........................................... ok.
disort BRDF AMBRALS file (0.00050,   0.1)...................................... ok.
twostr aerosol and water cloud (0.00001,   0.1)................................ ok.
twostrpp aerosol and water cloud (0.00001,   0.1).............................. ok.
c_twostr aerosol and water cloud (0.00001,   0.1).............................. ok.
rodents aerosol and water cloud (0.00001,   0.1)............................... ok.
rodents aerosol and water cloud, zout (0.00001,   0.1)......................... ok.
rodents aerosol and water cloud, zout, thermal (0.00001,   0.1)................ ok.
single scattering lidar, water cloud (0.00001,   0.1).......................... ok.
twostrebe aerosol and water cloud (0.00001,   0.1)............................. ok.
twomaxrand water cloud (0.00001,   0.1)........................................ ok.
twomaxrand wc/ic cloud (0.00001,   0.1)........................................ ok.
polradtran (0.00001,   0.1).................................................... ok.
profiles 1 (0.00001,   0.2).................................................... ok.
profiles 2 (0.00001,   0.1).................................................... ok.
profiles 3 (0.00001,   0.1).................................................... ok.
profiles 4 (0.00001,   0.1).................................................... ok.
radiosonde (0.00001,   0.4).................................................... ok.
redistribution (0.00001,   0.1)................................................ ok.
AVHRR [Kratz, 1995], channel 1, solar (0.00001,   0.1)......................... ok.
AVHRR [Kratz, 1995], channel 2, solar (0.00100,   0.1)......................... ok.
AVHRR [Kratz, 1995], channel 3, solar (0.00001,   0.1)......................... ok.
AVHRR [Kratz, 1995], channel 3, thermal (0.00001,   0.1)....................... ok.
AVHRR [Kratz, 1995], channel 4, thermal (0.00001,   0.1)....................... ok.
AVHRR [Kratz, 1995], channel 5, thermal (0.00001,   0.1)....................... ok.
correlated-k [Kato et al., 1999], twostr (0.00100,   0.1)...................... ok.
correlated-k, new Kato tables, twostr (0.00100,   0.1)......................... ok.
correlated-k [Fu and Liou, 1992/93], disort (0.01000,   1.5)................... ok.
correlated-k [Fu and Liou, 1992/93], ice clouds (0.01000,   1.5)............... ok.
Fu and Liou, thermal irradiance, disort (0.00001,   0.9)....................... ok.
LOWTRAN absorption parameterization (0.00001,   2.2)........................... ok.
LOWTRAN absorption parameterization, thermal (0.00001,   1.0).................. ok.
SSRadar Test (0.01000,   0.0).................................................. ok.
atmospheric reflectivity (0.00001,   0.1)...................................... ok.
IPA and correlated_k (0.00100,   0.2).......................................... ok.
cloudcover and correlated_k (0.00100,   1.8)................................... ok.
cloudcover, correlated_k and redistribution (0.00100,   1.8)................... ok.
wc_ipa_files and correlated_k (0.00100,   0.1)................................. ok.
wc_ipa_files, correlated_k and redistribution (0.00100,   0.1)................. ok.
wc_ipa_files, ic_ipa_files and redistribution (0.00001,   0.1)................. ok.
heating rates and wc_ipa_files (0.00100,   0.2)................................ ok.
cooling rates and wc_ipa_files (0.00100,   1.9)................................ ok.
transmittance_wl_file (0.00100,   0.1)......................................... ok.
raman (0.10000,   0.1)......................................................... ok.
fluorescence (0.10000,   0.1).................................................. ok.
reptran_thermal (0.01000,   0.1)............................................... ok.
reptran_solar (0.01000,   0.2)................................................. ok.
tzs (0.01000,   0.1)........................................................... ok.
disort_spherical_albedo (0.01000,   0.1)....................................... ok.
MYSTIC polarisation (0.00001,   3.0)........................................... ok.
MYSTIC polarized surface reflection (BPDF) (0.00001,   1.0)....................    1 serious differences.  Maximum difference:   1.030000%. Absolute values (test,ans): (-7.743240e-03, -7.663602e-03).
MYSTIC backward polarized surface reflection (BPDF) (0.00001,   1.0)........... ok.
MYSTIC spectral importance sampling (REPTRAN) (0.00001,   5.0)................. ok.
MYSTIC boxairmass factor (0.00001,   1.0)...................................... ok.
MYSTIC spherical (0.00001,   2.0).............................................. ok.
NCA 1.0 (0.00001,   0.1)....................................................... ok.
NCA 2.0 cuboid (0.00001,   0.1)................................................ ok.
NCA 2.0 triangle (0.00001,   0.1).............................................. ok.
1 of the \lstinline{uvspec} tests failed.
make[1]: Leaving directory '/home/dgriffith/libRadtran/libRadtran-2.0.3/test'
(devpy27) dgriffith@nimbus:~/libRadtran/libRadtran-2.0.3$
\end{lstlisting}

From the above test report it is evident that the \lstinline{MYSTIC} polarized surface reflection \ac{BPDF} test had a result error difference of 1.03\%.
All the other tests passed.

On the matter of tests failing in the check phase, see\\
\lstinline{http://www.libradtran.org/doku.php?id=faq}\\
where it states:\\
\textbf{How serious are serious differences reported by "make check"?}\\
They are usually not as serious as it sounds. We haven't had a case where libRadtran worked on one computer and produced really wrong results on another. However, radiative transfer equation solvers like disort are numerical methods which are affected by the limited numerical precisions of processors, and these differ from processor to processor and sometimes even from compiler to compiler. Usually, large differences occur for small numbers. As an example, in the near-infrared the diffuse downward irradiance is very small while the direct beam source is large - hence one may expect some uncertainty in the small diffuse radiation.

\end{enumerate}

\section{Introductory Example Use}

The following is taken (with some adaptation) from \\
\lstinline{http://www.libradtran.org/doku.php?id=basic_usage}

Those users who are not familiar with the predecessor of \libradtran(), \lstinline{uvspec}, please note the following: The central program of the package is an executable called \lstinline{uvspec} which can be found in the bin directory. If you are interested in a user-friendly program for radiative transfer calculations, \lstinline{uvspec} is the software you want to become familiar with. A description of \lstinline{uvspec} is provided in the first part of the manual. Examples of its use, including various input files and corresponding output files for different atmospheric conditions, are provided in the examples directory. For a quick try of \lstinline{uvspec} go to the examples directory and run \lstinline{uvspec}:

\begin{lstlisting}
    cd libRadtran-2.0.3/examples
    ../bin/uvspec < UVSPEC_CLEAR.INP > test.out
\end{lstlisting}



The input file for this example run is given in \lstinline{UVSPEC_CLEAR.INP}

\begin{lstlisting}[style=tinysize]
                         # Location of atmospheric profile file. 
atmosphere_file ../data/atmmod/afglus.dat
                         # Location of the extraterrestrial spectrum
source solar ../data/solar_flux/atlas_plus_modtran
mol_modify O3 300. DU    # Set ozone column
day_of_year 170          # Correct for Earth-Sun distance
albedo 0.2               # Surface albedo
sza 32.0                 # Solar zenith angle
rte_solver disort        # Radiative transfer equation solver
number_of_streams  6     # Number of streams
wavelength 299.0 341.0   # Wavelength range [nm]
slit_function_file ../examples/TRI_SLIT.DAT
                         # Location of slit function
spline 300 340 1         # Interpolate from first to last in step

quiet
\end{lstlisting}

This input file yields the following output file (the details of which you can decipher from the \libradtran{} User Guide \cite{libRadTranUserGuide2019}):

\begin{lstlisting}[style=tinysize]
  300.000  2.763049e+00  3.087059e+00  1.170022e+00  2.592736e-01  4.777781e-01  1.862147e-01 
  301.000  5.223602e+00  5.888303e+00  2.222381e+00  4.901621e-01  9.168083e-01  3.537029e-01 
  302.000  6.212306e+00  7.024819e+00  2.647425e+00  5.829381e-01  1.098937e+00  4.213508e-01 
  303.000  1.499798e+01  1.709326e+01  6.418247e+00  1.407351e+00  2.687637e+00  1.021496e+00 
  304.000  1.731212e+01  1.968946e+01  7.400318e+00  1.624501e+00  3.108258e+00  1.177797e+00 
  305.000  2.666450e+01  3.052658e+01  1.143822e+01  2.502091e+00  4.841135e+00  1.820449e+00 
  306.000  2.714423e+01  3.094949e+01  1.161874e+01  2.547107e+00  4.925832e+00  1.849181e+00 
  307.000  3.892420e+01  4.444759e+01  1.667436e+01  3.652493e+00  7.100925e+00  2.653807e+00 
  308.000  4.928456e+01  5.616349e+01  2.108961e+01  4.624668e+00  9.003217e+00  3.356516e+00 
  309.000  4.968203e+01  5.629452e+01  2.119531e+01  4.661965e+00  9.051084e+00  3.373338e+00 
  310.000  5.282494e+01  5.981413e+01  2.252781e+01  4.956883e+00  9.651869e+00  3.585413e+00 
  311.000  9.116537e+01  1.021252e+02  3.865812e+01  8.554597e+00  1.652179e+01  6.152631e+00 
  312.000  8.539021e+01  9.519187e+01  3.611642e+01  8.012679e+00  1.544426e+01  5.748106e+00 
  313.000  1.008859e+02  1.116138e+02  4.249995e+01  9.466736e+00  1.815837e+01  6.764078e+00 
  314.000  1.140307e+02  1.247804e+02  4.776222e+01  1.070019e+01  2.035162e+01  7.601594e+00 
  315.000  1.217260e+02  1.323822e+02  5.082165e+01  1.142229e+01  2.164887e+01  8.088517e+00 
  316.000  1.001420e+02  1.074433e+02  4.151704e+01  9.396925e+00  1.761202e+01  6.607643e+00 
  317.000  1.555369e+02  1.652335e+02  6.415407e+01  1.459496e+01  2.715088e+01  1.021044e+01 
  318.000  1.362324e+02  1.430189e+02  5.585025e+01  1.278350e+01  2.355464e+01  8.888844e+00 
  319.000  1.611656e+02  1.681876e+02  6.587064e+01  1.512314e+01  2.776797e+01  1.048364e+01 
  320.000  1.788111e+02  1.816904e+02  7.210032e+01  1.677893e+01  3.006627e+01  1.147512e+01 
  321.000  1.785449e+02  1.818189e+02  7.207275e+01  1.675394e+01  3.015396e+01  1.147073e+01 
  322.000  1.869200e+02  1.855010e+02  7.448420e+01  1.753983e+01  3.083632e+01  1.185453e+01 
  323.000  1.653979e+02  1.624324e+02  6.556606e+01  1.552028e+01  2.706200e+01  1.043516e+01 
  324.000  2.089290e+02  2.038225e+02  8.255030e+01  1.960507e+01  3.403920e+01  1.313829e+01 
  325.000  2.134198e+02  2.024978e+02  8.318353e+01  2.002647e+01  3.389131e+01  1.323907e+01 
  326.000  2.853074e+02  2.686987e+02  1.108012e+02  2.677211e+01  4.507038e+01  1.763456e+01 
  327.000  2.888047e+02  2.682587e+02  1.114127e+02  2.710029e+01  4.510012e+01  1.773188e+01 
  328.000  2.730194e+02  2.472749e+02  1.040589e+02  2.561906e+01  4.166240e+01  1.656148e+01 
  329.000  3.073699e+02  2.764369e+02  1.167614e+02  2.884237e+01  4.668488e+01  1.858315e+01 
  330.000  3.460204e+02  3.062719e+02  1.304585e+02  3.246919e+01  5.183652e+01  2.076311e+01 
  331.000  2.989884e+02  2.580901e+02  1.114157e+02  2.805589e+01  4.377979e+01  1.773236e+01 
  332.000  3.145912e+02  2.690056e+02  1.167194e+02  2.951999e+01  4.573355e+01  1.857646e+01 
  333.000  3.134522e+02  2.635762e+02  1.154057e+02  2.941311e+01  4.491315e+01  1.836738e+01 
  334.000  3.062941e+02  2.523394e+02  1.117267e+02  2.874142e+01  4.309492e+01  1.778186e+01 
  335.000  3.283018e+02  2.671566e+02  1.190917e+02  3.080654e+01  4.572950e+01  1.895403e+01 
  336.000  2.771408e+02  2.219277e+02  9.981371e+01  2.600579e+01  3.807083e+01  1.588585e+01 
  337.000  2.736382e+02  2.146446e+02  9.765657e+01  2.567713e+01  3.690510e+01  1.554253e+01 
  338.000  3.155172e+02  2.437741e+02  1.118583e+02  2.960688e+01  4.200866e+01  1.780280e+01 
  339.000  3.408671e+02  2.595133e+02  1.200761e+02  3.198561e+01  4.481992e+01  1.911070e+01 
  340.000  3.823216e+02  2.855006e+02  1.335645e+02  3.587555e+01  4.941755e+01  2.125744e+01 
\end{lstlisting}

This newly-calculated output file differs somewhat from the reference output file \lstinline{UVSPEC_CLEAR.OUT} present in the \lstinline{examples} folder.  The differences are small (fifth decimal), in the third column.  This is within the allowable variance limit for testing the installation.

\section{Status Review}

After the above installation there is a running \libradtran{} version 2.0.3 running on the server. The next step is to set up Dask to use this server. 