% -*- TeX -*- -*- UK -*- -*- Soft -*-


\chapter{Server Setup}
\label{chap:ServerSetup}



\section{Work on nimbus (146.64.246.94) for user \lstinline{dgriffith}}

Home directory \lstinline{~} is \lstinline{/home/dgriffith}

\subsection{Bring nimbus to Python 3 status}

Installed a python 3 environment for morticia: mordevpy37, with the morticia dependencies listed above.

Changed the folder name  \lstinline{~/GitHub/morticia} to \lstinline{~/GitHub/morticia2} to keep the old copy available.

Cloned the python 3 version of morticia into folder \lstinline{~/GitHub/morticia}, from \lstinline{https://github.com/NelisW/morticia}

To see where the \lstinline{site-packages} folder on your anaconda installation are use this code:

\begin{lstlisting}
    from distutils.sysconfig import get_python_lib
    print(get_python_lib())

\end{lstlisting}
Added \lstinline{morticia.pth} to the prepared environment's \lstinline{site-packages} folder, e.g.,  \lstinline{~/anaconda2/envs/mordevpy37/lib/python3.7/site-packages/morticia.pth}. The contents of this file is the single line. 

    \lstinline{/home/dgriffith/GitHub/morticia}


\subsection{libRadTran server}

1. On the PC activate the Python 3 morticia environment 

1. Start up the VPN (if used) and find the IP number allocated to the VPN ethernet0. On Windows uUse \lstinline{ipconfig} and look for something like the following (the Ethernet adapter connection number may be different, look for the DNS suffix that says \lstinline{csir.co.za}). This should be an IP address on the CSIR network \lstinline{146.64.xxx.xxx}:

\begin{lstlisting}
Ethernet adapter Local Area Connection 2:

Connection-specific DNS Suffix  . : csir.co.za
Link-local IPv6 Address . . . . . : fe80::8992:152a:c1f0:1260%27
IPv4 Address. . . . . . . . . . . : 146.64.202.118
Subnet Mask . . . . . . . . . . . : 255.255.0.0
Default Gateway . . . . . . . . . :
\end{lstlisting}

    To find the IP address on Linux use \lstinline{ip addr show}.


1. On the PC start the Dask scheduler, using the VPN IP address


\begin{lstlisting}
      dask-scheduler --host 146.64.202.118
\end{lstlisting}

1. On nimbus activate the Python 3 morticia environment 

1. On nimbus cd to the libRadTran bin folder  \lstinline{/home/dgriffith/libRadtran/libRadtran-2.0.2/bin/}

1. On nimbus start the Dask scheworker process, using the VPN IP address and the Dask port number

\begin{lstlisting}
    dask-worker tcp://146.64.202.118:8786
\end{lstlisting}

1. Open the notebook and set the scheduler using the VPN host address



