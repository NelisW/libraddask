% -*- TeX -*- -*- UK -*- -*- Soft -*-

\chapter{libraddask Setup}
\label{chap:libraddaskSetup}

\section{Set up Python for Dask}

\subsection{Creating the Python 3 environment}

Create an environment with Anaconda with the appropriate packages and versions to use for Dask.
In this case we are using Python 3.7 to run the Dask scheduler and workers on the server.
Exactly the same versions must be installed on the server and local PC.

On the local Windows computer:
\begin{lstlisting}
    conda create --name mort python=3.7 anaconda
\end{lstlisting}

On the remote Linux server:
\begin{lstlisting}
    conda create --name mordevpy37 python=3.7 anaconda
\end{lstlisting}


\subsection{Additional packages}
\label{sec:Additionalpackages}

Ensure that the following packages of the same version numbers are installed on all the computers. 
Some or all of these packages may already be present as part of the full Anaconda installation.
The package numbers installed at the current time on our computers are also listed.
Note that this list only covers the minimum Dask requirements, your local application may require other packages as well.
\begin{lstlisting}
    'dask': '2.12.0'
    'distributed': '2.12.0'
    'msgpack': '0.6.1'
    'cloudpickle': '1.3.0'
    'tornado': '6.0.4'
    'toolz': '0.10.0'
    'numpy': '1.18.1'
\end{lstlisting}

Search for packages here: 
\lstinline{https://anaconda.org/anaconda/} for standard Anaconda packages and 
\lstinline{https://anaconda.org/conda-forge/} for community-supported packages.

Note that \lstinline{msgpack} is known in conda as \lstinline{msgpack-python}.

To see a list of the currently installed packages on a computer type one of the following, to see versions for all package or a specific package:
\begin{lstlisting}
conda list
conda list dask
\end{lstlisting}

To install a specific version of a package use the command, with appropriate replacement of the package name and version number:
\begin{lstlisting}
conda install package=version

conda install dask=2.12.0
conda install distributed=2.12.0
conda install msgpack-python=0.6.1
\end{lstlisting}


\subsection{Install \libraddask{}}
\label{sec:Installlibraddask}

The \libraddask{} library is normally cloned from the GitHub repository and not installed via PyPi or conda. 
Create a folder into which \libraddask{} can be cloned. In this case assume the folder on a Windows computer is \lstinline{V:\work\WorkN}.  Change the current working directory to this newly-created folder and clone the \libraddask{}  repository from GitHub. In a command window do:
\begin{lstlisting}
v:
cd \work\WorkN
git clone https://github.com/NelisW/libraddask.git
\end{lstlisting}
This should clone \libraddask{}  into the \lstinline{V:\work\WorkN\libraddask} folder.

To tell Python where to find this library create a file with the filename \lstinline{\libraddask.pth} in the \lstinline{site-packages} folder of the same environment where dask present. When Python is searching for the location of the \libraddask{} module, it will find this file, which will indicate the location where the module is installed.
To see where the \lstinline{site-packages} folder for the dask conda environment is,  activate the environment and execute this code:
\begin{lstlisting}
    from distutils.sysconfig import get_python_lib
    print(get_python_lib())
\end{lstlisting}

On my\textbf{ Windows} computer the file is saved in the environment here: 
\begin{lstlisting}
    C:/ProgramData/Anaconda3/envs/mort/lib/site-packages\libraddask.pth
\end{lstlisting}
The \lstinline{\libraddask.pth} file must have only one line, and this line must be the \libraddask{}  library location.    In my case the library was cloned from the repository into the following folder, and this must be the contents of the file:
\begin{lstlisting}
    V:\work\WorkN
\end{lstlisting}

On the \textbf{Linux} computer (the nimbus server in this case)  the file is saved in the environment here: 
\begin{lstlisting}
/home/dgriffith//anaconda2/envs/mordevpy37/lib/python3.7/site-packages/libraddask.pth
\end{lstlisting}
The file contents must be a full path, the `$\sim$/libraddask` form does not seem to work:
\begin{lstlisting}
/home/dgriffith/libraddask
\end{lstlisting}

\subsection{Prepare for Jupyter}
If the jupyter notebook will be used in your work,  install the Jupyter notebook in Python.
\begin{lstlisting}
conda install notebook ipykernel
ipython kernel install
\end{lstlisting}
See here for more detail:\\
\lstinline{https://github.com/NelisW/ComputationalRadiometry/blob/master/00-Installing-Python-pyradi.ipynb}.

