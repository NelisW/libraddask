\chapter{Porting \texttt{scr\_py} to Python 3}
\label{chap:Portingscr_pytoPython3}

\section{Original Work}
\label{sec:portingAppOriginalWork}

Derek Griffith's original work was done with the Python files for \libradtran{} 2.0.0.
This work will therefore update from 2.0.0 to 2.0.3 as well as Python 3 porting.  Griffith's additions were as follows:
\begin{enumerate}
\item \lstinline{aerosol_options.py}: a number of changes:

\begin{lstlisting}
--- W:\temp\libradold\200\aerosol_options.py 2020-04-21  10:34:12
+++ W:\temp\libradold\DerekGitHubMORTICIAmorticiarad\aerosol_options.py 2020-04-21  11:01:58
@@ -23,28 +23,30 @@
  * Boston, MA 02111-1307, USA.
  *--------------------------------------------------------------------"""
 
+# Modified by Derek Griffith to include aerosol_sizedist_file and aerosol_refrac_index
+
 
 from option_definition import *
 
@@ -205,216 +207,238 @@
 			extra_dependencies=['aerosol_haze','aerosol_vulcan','aerosol_visibility','aerosol_season'],
 		)
 	
+        aerosol_sizedist_file = option(
+            name='aerosol_sizedist_file',
+            group='aerosol',
+            helpstr='Aerosol size distribution file.',
+            documentation=documentation['aerosol_sizedist_file'],
+            tokens=addToken(name='Input.aer.filename[Id]', datatype=file),
+            childs=['aerosol_option_specification'],
+            extra_dependencies=[],
+        )
+        aerosol_refrac_index = option(
+            name='aerosol_refrac_index',
+            group='aerosol',
+            helpstr='Set the aerosol particle refractive index.',
+            documentation=documentation['aerosol_refrac_index'],
+            tokens=[addToken(name='Input.aer.nreal', datatype=float),
+                    addToken(name='Input.aer.nimag', datatype=float, default='NOT_DEFINED_FLOAT', valid_range=[0, 1e6]),
+                    addSetting(name='Input.aer.spec', setting=1)],
+            parents=['aerosol_default'],
+        )
 		self.options = [ aerosol_default,
 			aerosol_file, aerosol_species_library, aerosol_species_file, 
 			aerosol_haze, aerosol_season, aerosol_vulcan, aerosol_visibility,
 			aerosol_profile_modtran,
 			aerosol_angstrom,aerosol_king_byrne,
-			aerosol_modify, aerosol_set_tau_at_wvl ]
+                        aerosol_modify, aerosol_set_tau_at_wvl, aerosol_sizedist_file,
+                        aerosol_refrac_index]
 
 	def __iter__(self):
 		return iter(self.options)
@@ -493,498 +515,536 @@
 	\fcode{
 	   aerosol\_set\_tau\_at\_wvl lambda tau
 	}
+		''',
+
+        'aerosol_sizedist_file': r'''
+        Calculate optical properties from size distribution and index of refraction using Mie
+        theory. Here is an exception from the rule that ALL values defined above are overwritten
+        because the optical thickness profile is re-scaled so that the optical thickness
+        at the first internal wavelength is unchanged. It is done that way to give the user an
+        easy means of specifying the optical thickness at a given wavelength.
+        ''',
+
+        'aerosol_refrac_index': r'''
+        Calculate optical properties from size distribution and index of refraction using Mie
+        theory. Here is an exception from the rule that ALL values defined above are overwritten
+        because the optical thickness profile is re-scaled so that the optical thickness
+        at the first internal wavelength is unchanged. It is done that way to give the user an
+        easy means of specifying the optical thickness at a given wavelength.
 		''',
 }
\end{lstlisting}
 


\item \lstinline{spectral_options.py}: added \lstinline{wavelength_step} option:

\begin{lstlisting}
--- W:\temp\libradold\200\spectral_options.py 2020-04-21  10:34:14
+++ W:\temp\libradold\DerekGitHubMORTICIAmorticiarad\spectral_options.py 2016-09-23  11:33:02
@@ -46,51 +46,62 @@
 			non_parents=['wavelength_index'],
 		)
 
+		wavelength_step = option(
+			name='wavelength_step',
+			group='spectral',
+			helpstr='Set the wavelength step (in nm) in conjunction with the wavelength range.',
+			documentation=['wavelength_step'],
+			gui_inputs=(),
+			tokens=[],
+			parents=['uvspec'],
+			non_parents=['wavelength_index'],
+		)
+
 		wavelength_index = option(
 			name='wavelength_index',
 			group='spectral',
\end{lstlisting}
\end{enumerate}



\section{Porting}
\label{sec:Portingscr_pytoPython3}

The following work was done:
\begin{enumerate}
\item Change all print statement lines to print functions.

\item Python 3 changed the local import syntax. The \libradtran{} files are in a separate folder, which requires am awkward means to specify the folder.
     All local imports must be of the form from \lstinline{libraddask.libradtran import option_definition}. 
     
     I also opted to not do a star import, which means that you must provide `a path' such as 
     \lstinline{option_definition.option} instead of the starred import version \lstinline{option}.  This change required a large number of changes throughout the code.

\item Updated Python 2 \lstinline{file} type tests to Python 3 \lstinline{io.IOBase} tests. For more detail see:\\
\lstinline{https://docs.python.org/3/library/io.html#io.IOBase},\\ ``the abstract base class for all I/O classes, acting on streams of bytes.''

\item
Changed exceptions:
\lstinline{except Exception, e:} to 
\lstinline{except Exception as e:}

\item 
Changed data type from \lstinline{double} to \lstinline{float}.\\
Changed data type from \lstinline{long} to \lstinline{int}.


\item 
The following files were changed with the original additions by Derek Griffith (see above):
\lstinline{aerosol_options.py}, \lstinline{spectral_options.py}.  These changes will be applied after all the other Python conversion modifications.


\end{enumerate}


\section{New Code Testing}

The following tests were done on revision 
\lstinline{678c42400523b7d730d6646132e271a7cacd8ab4}.
These tests are not exhaustive or very deep. The focus was more on determining if the porting to Python~3 did not cause breaking changes.

\begin{enumerate}
\item \lstinline{libraddask_test.py}.  The libRadtran files were scanned and all possible options that can be set.  The test entails creating an \lstinline{empty libraddask.rad.librad.Case} instance and then setting the options available by \lstinline{set_option()} observing the resulting \lstinline{case} output.  Some options coded in the files were not available and such \lstinline{set-option()} lines were commented out.  The data used in this test are not representative of values that have any physical significance, do this test must not be seen as a tutorial on how to use.

\item 
The notebook shown in Section~\ref{chap:ServerUserManual} was executed and checked for correct operation.  It executed all required functionality.

One issue arose: the \libradtran{} results different from transmittance results calculated for an atmosphere with a given set of Angstrom coefficients.  This is not a package issue, it is a \libradtran{} issue, still to be resolved.

\item
A number of internal (proprietary) notebooks were executed against this version of the package and all notebooks completed successfully. 


\end{enumerate}
