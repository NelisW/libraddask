% -*- TeX -*- -*- UK -*- -*- Soft -*-


\chapter{Server User Manual}
\label{chap:ServerUserManual}

\begin{lstlisting}

## Running libRadtran

### Additional data files
 
The libRadtran run below assumes that the following files are present in the version of libRadtran run. The text files must have Linux line endings.

    examples/moltauKLines56fa5f13.nc
    data/solar_flux/kurucz_full_749nm_770nm.dat
    data/solar_flux/kurucz_full_757nm_773nm.dat
    data/solar_flux/kurucz_full_760nm_777nm.dat

### ipyparallel

It is proposed to rather use Dask.

### Dask

Setting up the Dask environment on the nimbus server requires the following steps

See the libraddask report for instructions on how to set up the server environment: installing libRadtran and Python with dask.

The instructions here describe the use of the OSS nimbus server (146.64.246.94).

1. If working off-site, use a VPN client to connect to the same network as used by the server.  Dask requires that your local PC must have an IP number in the same subnetwork as the server.  Some VPNs do not provide an IP number on the same subnetwork and such VPN clients will not work with Dask.

1. Using a remote terminal client, open three terminal windows on the server: (1) a  terminal for general use, (2) a terminal for the Dask scheduler, and (3) a terminal for the Dask workers.

1. In the scheduler terminal activate the conda environment with the Dask packages and then start the scheduler.

        conda activate mordevpy37
        dask-scheduler
        
   This should start the scheduler:
   
        (base) dgriffith@nimbus:~$ conda activate mordevpy37
        (mordevpy37) dgriffith@nimbus:~$ dask-scheduler
        distributed.scheduler - INFO - -----------------------------------------------
        distributed.scheduler - INFO - Local Directory:    /tmp/scheduler-ay08oqkc
        distributed.scheduler - INFO - -----------------------------------------------
        distributed.scheduler - INFO - Clear task state
        distributed.scheduler - INFO -   Scheduler at:  tcp://146.64.246.94:8786
        distributed.scheduler - INFO -   dashboard at:                     :8787

    Observe the scheduler URI: `tcp://146.64.246.94:8786`, this must be used in the next step to set up the workers.
    
1. In the worker terminal change to the `bin` folder in the libRadtran installation folder and activate the conda environment with the Dask packages and then start the worker, using the scheduler URI and setting the required number of processors on the cluster :

        cd libRadtran/libRadtran-2.0.3/bin    
        conda activate mordevpy37    
        dask-worker tcp://146.64.246.94:8786    
        
    This should start the workers, each with a different port number (not all detail shown):
    
        (base) dgriffith@nimbus:~$ cd libRadtran/libRadtran-2.0.3/bin
        (base) dgriffith@nimbus:~/libRadtran/libRadtran-2.0.3/bin$ conda activate mordevpy37
        (mordevpy37) dgriffith@nimbus:~/libRadtran/libRadtran-2.0.3/bin$ dask-worker tcp://146.64.246.94:8786 --nprocs 8
        distributed.nanny - INFO -         Start Nanny at: 'tcp://146.64.246.94:33671'
        distributed.nanny - INFO -         Start Nanny at: 'tcp://146.64.246.94:46725'
        distributed.nanny - INFO -         Start Nanny at: 'tcp://146.64.246.94:42509'
        distributed.nanny - INFO -         Start Nanny at: 'tcp://146.64.246.94:42983'
        distributed.nanny - INFO -         Start Nanny at: 'tcp://146.64.246.94:43735'
        distributed.nanny - INFO -         Start Nanny at: 'tcp://146.64.246.94:36947'
        distributed.nanny - INFO -         Start Nanny at: 'tcp://146.64.246.94:40075'
        distributed.nanny - INFO -         Start Nanny at: 'tcp://146.64.246.94:40495'
        ...
        distributed.worker - INFO -       Start worker at:  tcp://146.64.246.94:36533
        distributed.worker - INFO -          Listening to:  tcp://146.64.246.94:36533
        distributed.worker - INFO -          dashboard at:        146.64.246.94:43069
        distributed.worker - INFO - Waiting to connect to:   tcp://146.64.246.94:8786
        distributed.worker - INFO - -------------------------------------------------
        distributed.worker - INFO -               Threads:                          4
        distributed.worker - INFO -                Memory:                    4.21 GB
        distributed.worker - INFO -       Local Directory: /home/dgriffith/libRadtran/libRadtran-2.0.3/bin/dask-worker-space/worker-_0o0qta1
        distributed.worker - INFO - -------------------------------------------------
        distributed.worker - INFO -         Registered to:   tcp://146.64.246.94:8786
        distributed.worker - INFO - -------------------------------------------------
        distributed.core - INFO - Starting established connection
        ...
        distributed.worker - INFO -       Start worker at:  tcp://146.64.246.94:46597
        distributed.worker - INFO -          Listening to:  tcp://146.64.246.94:46597
        distributed.worker - INFO -          dashboard at:        146.64.246.94:33737
        distributed.worker - INFO - Waiting to connect to:   tcp://146.64.246.94:8786
        distributed.worker - INFO - -------------------------------------------------
        distributed.worker - INFO -               Threads:                          4
        distributed.worker - INFO -                Memory:                    4.21 GB
        distributed.worker - INFO -       Local Directory: /home/dgriffith/libRadtran/libRadtran-2.0.3/bin/dask-worker-space/worker-n0anjq00
        distributed.worker - INFO - -------------------------------------------------
        distributed.core - INFO - Starting established connection
        distributed.worker - INFO -         Registered to:   tcp://146.64.246.94:8786
        distributed.worker - INFO - -------------------------------------------------
        distributed.core - INFO - Starting established connection


1. Run the code that activates the client in the local PC (see below). This should send the serialised data to the scheduler:

        distributed.scheduler - INFO - Register worker <Worker 'tcp://146.64.246.94:33437', name: tcp://146.64.246.94:33437, memory: 0, processing: 0>
        distributed.scheduler - INFO - Starting worker compute stream, tcp://146.64.246.94:33437
        distributed.core - INFO - Starting established connection
        distributed.scheduler - INFO - Receive client connection: Client-1ed95fc0-8076-11ea-8700-cfd8862f98eb
        distributed.core - INFO - Starting established connection
        distributed.scheduler - INFO - Receive client connection: Client-27b067b0-8076-11ea-8700-cfd8862f98eb
        distributed.core - INFO - Starting established connection

    The scheduler will send the serialised data to the workers, which will execute the tasks:
    
        distributed.core - INFO - Event loop was unresponsive in Worker for 1.29s.  This is often caused by long-running GIL-holding functions or moving large chunks of data. This can cause timeouts and instability.
        Current working directory: "/home/dgriffith/libRadtran/libRadtran-2.0.3/bin"
        Current working directory: "/home/dgriffith/libRadtran/libRadtran-2.0.3/bin"
        distributed.core - INFO - Event loop was unresponsive in Worker for 1.20s.  This is often caused by long-running GIL-holding functions or moving large chunks of data. This can cause timeouts and instability.
        Current working directory: "/home/dgriffith/libRadtran/libRadtran-2.0.3/bin"
        distributed.core - INFO - Event loop was unresponsive in Worker for 1.21s.  This is often caused by long-running GIL-holding functions or moving large chunks of data. This can cause timeouts and instability.
        Current working directory: "/home/dgriffith/libRadtran/libRadtran-2.0.3/bin"


        
If the libRadtran execution was successful the tasks should complete and the data returned to the client.


\section{Jupyter Notebook Template}



import libraddask.rad.librad as librad

from dask.distributed import Client # For contacting the dask scheduler

serverusername = '146.64.246.94:8786'

# dask Client method with scheduler IP and port
paraclient = Client(serverusername)

futureRadBatch = paraclient.map(librad.Case.run, [S3, S3b])
# Gather results. This will wait for completion of all tasks.
S3List = paraclient.gather(futureRadBatch)  
\end{lstlisting}